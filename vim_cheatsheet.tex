\DocumentMetadata{
    lang        = {en-US},
    pdfversion  = 2.0,
    pdfstandard = ua-2,
    pdfstandard = a-4f,
    testphase   = {
        phase-III,
        title,
        table,
        % math,
        firstaid
    }
}

\documentclass[a4paper, landscape, 10pt]{article}
\frenchspacing
\setlength{\parskip}{0pt}\raggedbottom
\raggedright

\PassOptionsToPackage{babel}{main=english, math=normal, showlanguages}

\usepackage{iftex}

\ifluatex
    \pdfvariable omitcidset=1
\fi

\usepackage{babel}

\usepackage{calc}
\usepackage{geometry}
% \PassOptionsToPackage{textpos}{absolute}
% \usepackage{textpos}

\usepackage{graphicx}
% \usepackage{pdfpages}
% \usepackage{svg}
\PassOptionsToPackage{xcolor}{rgb,hyperref}
\usepackage{xcolor}
% \ifluatex
%     \usepackage{luacolor}
% \fi

% \usepackage{alltt}
% \usepackage{array}
% \usepackage{booktabs}
% \usepackage{caption}
% \usepackage{enumitem}\setlist{nosep}
% \usepackage{enumerate}
% \usepackage{paralist}
% \usepackage{fancyhdr}
% \usepackage{float}
% \usepackage{footmisc}
% \usepackage{framed}
\usepackage{multicol}
\ifluatex
    \usepackage{realscripts}
\fi
\usepackage[toc,eqno,enum,bib,lineno]{tabfigures}
\usepackage{tabularx}
% \usepackage{titlesec}
% \usepackage{tocloft}

\usepackage[tbtags]{mathtools}
\usepackage{amsthm}

\usepackage{unicode-math}
\defaultfontfeatures{Renderer = OpenType}
\usepackage{microtype}

\usepackage{hyperref}
\usepackage{cleveref}


\newlength{\gutter}
\setlength{\gutter}{2.7cm}
\newlength{\shift}
\setlength{\shift}{.0cm}
\newlength{\mycolsep}
\setlength{\mycolsep}{.65cm}
\newlength{\lmargin}
\setlength{\lmargin}{.5\gutter - \mycolsep + \shift}
\newlength{\rmargin}
\setlength{\rmargin}{.5\gutter - \mycolsep - \shift}
\geometry{lmargin=\lmargin, rmargin=\rmargin, vmargin=1cm}

\definecolor{solarizedbase03}{HTML}{002b36}
\definecolor{solarizedbase02}{HTML}{073642}
\definecolor{solarizedbase01}{HTML}{586e75}
\definecolor{solarizedbase00}{HTML}{657b83}
\definecolor{solarizedbase0}{HTML}{839496}
\definecolor{solarizedbase1}{HTML}{93a1a1}
\definecolor{solarizedbase2}{HTML}{eee8d5}
\definecolor{solarizedbase3}{HTML}{fdf6e3}
\definecolor{solarizedyellow}{HTML}{b58900}
\definecolor{solarizedorange}{HTML}{cb4b16}
\definecolor{solarizedred}{HTML}{dc322f}
\definecolor{solarizedmagenta}{HTML}{d33682}
\definecolor{solarizedviolet}{HTML}{6c71c4}
\definecolor{solarizedblue}{HTML}{268bd2}
\definecolor{solarizedcyan}{HTML}{2aa198}
\definecolor{solarizedgreen}{HTML}{859900}

\babelfont{rm}[
    Numbers = {Uppercase, Proportional},
    Scale = 0.9,
    ScaleAgain = 1.0,
    Path = /home/marc/Downloads/FONTS/Adobe Source/variable/,
    Extension = .ttf,
    UprightFont = *VF,
    UprightFeatures = { RawFeature = {axis={wght=400}} },
    ItalicFont = *VF-Italic,
    ItalicFeatures = { RawFeature = {axis={wght=400}} },
    BoldFont = *VF,
    BoldFeatures = { RawFeature = {axis={wght=700}} },
    BoldItalicFont = *VF-Italic,
    BoldItalicFeatures = { RawFeature = {axis={wght=700}} },
    FontFace = {ub}{n}{Font = *VF, RawFeature = {axis={wght=900}}},
    FontFace = {ub}{it}{Font = *VF-Italic, RawFeature = {axis={wght=900}}},
]{SourceSans3}

\babelfont{tt}[
    Scale = MatchUppercase,
    ScaleAgain = 1.0,
    Path = /home/marc/Downloads/FONTS/JetBrains Mono/variable/,
    Extension = .ttf,
    UprightFont = *VF,
    UprightFeatures = { RawFeature = {axis={wght=400}} },
    ItalicFont = *VF-Italic,
    ItalicFeatures = { RawFeature = {axis={wght=400}} },
    BoldFont = *VF,
    BoldFeatures = { RawFeature = {axis={wght=700}} },
    BoldItalicFont = *VF-Italic,
    BoldItalicFeatures = { RawFeature = {axis={wght=700}} },
    Colour = solarizedbase01,
]{JetBrainsMono}

\makeatletter
\renewcommand{\section}{%
    \@startsection{section}%
        {1}%
        {0pt}%
        {-15pt}%
        {5pt}%
        {\fontsize{14pt}{12pt}\fontseries{ub}\selectfont\color{accentcolor}}}
\renewcommand{\subsection}{%
    \@startsection{subsection}%
        {2}%
        {0pt}%
        {-10pt}%
        {5pt}%
        {\fontsize{12pt}{12pt}\selectfont\bfseries\itshape\color{accentcolor}}}
\makeatother

\setlength{\columnsep}{\mycolsep}

\definecolor{shadecolor}{named}{solarizedbase3}

\pagestyle{empty}\thispagestyle{empty}

\newcommand{\V}[1]{\texttt{\textup{#1}}}

\newcommand{\topic}[1]{\section*{\protect\makebox(0,0)[l]{\color{accentcolor!80}\rule[8pt]{\columnwidth}{19pt}}\hspace{.5em}\color{white}#1}}
% \newcommand{\topic}[1]{\section*{\protect\makebox(0,0)[l]{\color{accentcolor!20}\rule[9pt]{\columnwidth}{19pt}}\hspace{.5em}#1}}
\newcommand{\subtopic}[1]{\subsection*{\protect\makebox(0,0)[l]{\color{accentcolor!20}\rule[7pt]{\columnwidth}{19pt}}\hspace{.5em}#1}}
% \newcommand{\subtopic}[1]{\subsection*{\protect\makebox(0,0)[l]{\color{accentcolor!10}\rule[7pt]{\columnwidth}{15pt}}\hspace{.5em}#1}}

\hypersetup{%
    pdftitle={Vim quick reference},
    pdfauthor={drs-p},
    pdfdate={2023-09-19},
    pdfkeywords={vim},
    pdfsubject={A quick reference for Vim},
    pdfcopyright={\copyright\ 2020–2023 drs-p. Released under the terms of the CC-BY-SA 4.0 license.},
}

\title{\GetDocumentProperties{hyperref/pdftitle}}
\author{\GetDocumentProperties{hyperref/pdfauthor}}
\date{\GetDocumentProperties{hyperref/pdfdate}}


\begin{document}
\begin{multicols*}{3}
    {
        {
            \fontseries{ub}\selectfont
            \definecolor{shadecolor}{HTML}{ffffff}%
            \hspace{.02\columnwidth}%
            \resizebox{.94\columnwidth}{!}{%
                \textcolor{solarizedbase01}%
                {\GetDocumentProperties{hyperref/pdftitle}}}
        }
%
        \color{solarizedbase00}\noindent
        \makebox[.955\columnwidth]
            {\mdseries\footnotesize\quad \emph{by} drs-p\hfill\emph{version} \GetDocumentProperties{hyperref/pdfdate}}
        \vspace*{-.2\baselineskip}
    }



\definecolor{accentcolor}{named}{solarizedgreen}
    \topic{Moving around}
    \begin{tabularx}{\columnwidth}{l>{\raggedright\arraybackslash}X}
\V{gg}, \V{G}
        &Go to top/end of file (with prefix \textit{n:} go to line \textit{n})\\
\verb|^e|, \verb|^d|, \verb|^f|
        &Scroll down one line/half a screen/full screen\\
\verb|^y|, \verb|^u|, \verb|^b|
        &Scroll up one line/half a screen/full screen\\
\V{H}, \V{M}, \V{L}
        &Go to top/middle/bottom of screen\\
\V{zt}, \V{z<Enter>}
        &Scroll current line to top of screen\\
\V{zz}, \V{z.}
        &Scroll current line to center of screen\\
\V{zb}, \V{z-}
        &Scroll current line to bottom of screen\\
        &(prefix \textit{n:} scroll line \textit{n} to top/center/bottom)\\
\verb|^o|, \verb|^i|
        &Go to location of last/next change or jump
    \end{tabularx}

\subtopic{Marks}
    \begin{tabularx}{\columnwidth}{l>{\raggedright\arraybackslash}X}
\V{mx}
        &Set mark \V{x} (\V{a} – \V{z} per buffer, \V{A} – \V{Z} across buffers)\\
\V{:marks}
        &List all current marks\\
\V{'x}
        &Jump to (begin of line containing) mark \V{x}\\
\V{`x}
        &Jump to exact location of mark \V{x}\\
\V{''}
        &Jump back to start of last jump
    \end{tabularx}



\definecolor{accentcolor}{named}{solarizedcyan}
\topic{Insert mode}
An optional prefixed count indicates how many characters or lines to insert or replace.
\vspace{.25\baselineskip}

    \begin{tabularx}{\columnwidth}{l>{\raggedright\arraybackslash}X}
\V{i}, \V{a}
        &Insert at/after cursor position\\
\V{gI}, \V{I}, \V{A}
        &Insert at start/begin/end of line\\
\V{o}, \V{O}
        &Insert new line after/before current line\\
\V{s}
        &Delete char(s) and start Insert mode\\
\V{C}
        &Delete rest of line, start Insert mode\\
\V{S}
        &Delete entire line, start Insert mode\\
\V{<esc>}, \verb|^[|
        &Return to Normal mode
    \end{tabularx}

\subtopic{Insert mode commands}
    \begin{tabularx}{\columnwidth}{l>{\raggedright\arraybackslash}X}
\verb|^d|, \verb|^t|
        &Shift left/right (by shiftwidth)\\
\verb|^k xy|
        &Insert accented characters (\V{:digraphs} for full list)

        \verb|a`| = à, \verb|a'| = á, \verb|a:| = ä, \verb|c,| = ç, \verb|n~| = ñ, \verb|ss| = ß, \V{>{}>} = »\\
\verb|^v k|
        &Key \textit{k} (e.g., <Esc>, <Enter>) is included verbatim.\linebreak
        If \textit{k} = \V{o777}, \V{999}, \V{xFF}, \V{uFFFF}, \V{UFFFFFFFF}: Unicode char\\
\verb|^o|
        &Execute single command, then return to Insert mode\\
\verb|^r r|
        &Insert contents of register \verb|r| (see \emph{Registers})\\
\verb|^a|, \verb|^@|
        &Repeat previous insert (\verb|^@| terminates Insert mode)
    \end{tabularx}



\definecolor{accentcolor}{named}{solarizedblue}
\topic{Text editing}
General editing actions are of the form:

{
    \vspace*{.5\baselineskip}
    \ttfamily\qquad [register] operator [count] operand
    \vspace*{.5\baselineskip}
}

``Operand'' may be a motion, text object, visual selection, search or mark.
As a shortcut, operators can be doubled to work on full lines:
\V{cc}, \V{dd}, \V{yy}, \V{<{}<}, \V{>{}>}, \V{==}, \V{guu}, \V{gUU}, \verb|g~~| etc.
Naming a specific register with operators \verb|d|, \verb|y|, \verb|p| yanks to or pastes from that register.

\subtopic{Operators}
    \begin{tabularx}{\columnwidth}{l>{\raggedright\arraybackslash}X}
\V{d}
        &Delete text\\
\V{y}
        &Copy (``yank'') text\\
\V{c}
        &Change text\\
\verb|g~|
        &Swap case\\
\V{gu}, \V{gU}
        &Make lower/uppercase\\
\V{<}, \V{>}, \V{=}
        &Shift left/right (by \V{shiftwidth}), autoindent\\
\V{gq}, \V{gw}
        &Format text; \V{gq} uses \V{formatexpr}/\V{formatprg}
    \end{tabularx}

\subtopic{Motions}
Motions move the cursor when used without operator.
\vspace{.25\baselineskip}

    \begin{tabularx}{\columnwidth}{l>{\raggedright\arraybackslash}X}
\V{h}, \V{j}, \V{k}, \V{l}
        &Left, down, up, right (\V{gj}, \V{gk} move by \emph{screen} line)\\
\verb|0|, \verb|^|, \verb|$|
        &Go to start/begin/end of line\\
        &(``begin'' is first non-space character of line)\\
\V{w}, \V{W}
        &Go to next word/WORD (``WORDS'' delimited only by whitespace; ``words'' also by punctuation)\\
\V{b}, \V{B}
        &Go to previous word/WORD\\
\V{e}, \V{ge}, \V{E}, \V{gE}
        &Go to end of next/previous word/WORD\\
\V{\%}
        &Go to matching parenthesis/brace\\
\verb|(|, \verb|)|
        &Go to beginning of current/next sentence\\
\V{\{}, \V{\}}
        &Go to beginning of current/next paragraph
    \end{tabularx}

\subtopic{Text objects}
These \V{a}~commands include surrounding whitespace; alternative \V{i}~forms don't. As a general rule, use \V{a}~forms with delete,\\\V{i}~forms with copy and change (mnemonic: \emph{around} versus \emph{inside}).
\vspace{.25\baselineskip}

    \begin{tabularx}{\columnwidth}{l>{\raggedright\arraybackslash}X}
\V{aw}, \V{aW}, \V{as}, \V{ap}
        &Word, WORD, sentence, paragraph\\
\V{a)}, \V{a]}, \V{a\}}, \V{a>}
        &Block in \V{()}, \V{[]}, \V{\{\}}, \V{<>} (\V{ab}, \V{aB} same as \V{a)}, \V{a\}})\\
\V{a"}, \V{a'}, \V{a`}, \V{at}
        &Various strings; HTML/XML tag
    \end{tabularx}

\subtopic{Visual mode}
Use motions, search, text objects or marks to select text.
\vspace{.25\baselineskip}

    \begin{tabularx}{\columnwidth}{l>{\raggedright\arraybackslash}X}
\verb|v|, \verb|V|, \verb|^v|
        &Select characters/lines/block
    \end{tabularx}

\subtopic{Search and replace}
    \begin{tabularx}{\columnwidth}{l>{\raggedright\arraybackslash}X}
\V{fx}, \V{tx}, \V{Fx}, \V{Tx}
        &Find/move to next/previous \V{x} on current line\\
\V{;} \V{,}
        &Repeat find in same/oposite direction\\
\V{/pat}, \V{?pat}
        &Search next/prev occurrence of pattern\\
\V{*}, \verb|#|
        &Search next/prev occurrence of current word\\
\V{n}, \V{N}
        &Repeat search in same/opposite direction\\
\V{:s/old/new/}
        &Change first \V{old} on current line to \V{new}
        \linebreak Flags: \V{g} = change all; \V{c} = confirm;
        \linebreak \V{e} = no error if \V{old} not found
        \\
\V{:\%s/old/new/g}
        &Change all \V{old} in current buffer to \V{new}
    \end{tabularx}

\subtopic{Miscellaneous other editing commands}
These command may also be prefixed with a count.
\vspace{.25\baselineskip}

    \begin{tabularx}{\columnwidth}{l>{\raggedright\arraybackslash}X}
\V{C}, \V{D}
        &Change/delete rest of current line (same as \verb|c$|, \verb|d$|)\\
\V{S}, \V{Y}
        &Substitute (change)/yank \emph{entire} current line\\
\V{p}, \V{P}
        &Paste yanked text after/before cursor\\
\V{]p}, \V{[P}
        &Same as \V{p}/\V{P}, but match current indentation\\
\V{x}, \V{X}
        &Delete character under/before cursor\\
\V{rx}
        &Replace character under cursor with \V{x}\\
\V{R}
        &Replace mode (overwrite existing text until \V{<Esc>})\\
\verb|~|
        &Change case of character under cursor\\
\V{J}, \V{gJ}
        &Join current and next lines, with/without space\\
\verb|^a|, \verb|^x|
        &Increment/decrement number under cursor
    \end{tabularx}

\subtopic{Repeat and undo}
    \begin{tabularx}{\columnwidth}{l>{\raggedright\arraybackslash}X}
\verb|.|
        &Repeat last command\\
\verb|u|
        &Undo last operation\\
\verb|^r|
        &Redo ``undone'' operation\\
\verb|U|
        &Undo all edits on current line (when cursor still on line)
    \end{tabularx}



\definecolor{accentcolor}{named}{solarizedviolet}
\topic{Registers and macros}
    \begin{tabularx}{\columnwidth}{l>{\raggedright\arraybackslash}X}
\V{""}
        &``Unnamed'' (default) register\\
\V{"a} – \V{"z}
        &Registers; use with \V{d}, \V{y}, \V{p} (\V{"A} – \V{"Z} \emph{append} to \V{a} – \V{z})\\
\V{"0}
        &Contains last yanked text\\
\V{"1} – \V{"9}
        &Stack containing ``big'' deletes (one or more lines)\\
\V{"-}
        &Last ``small'' delete (only part of line)\\
\V{"+}
        &System clipboard (if supported; check \V{:version})\\

\V{qr}, \V{qR}
        &Record operations into register \V{r} (\V{qR}: \emph{append} to \V{r})\\
\V{q}
        &Quit recording\\
\V{@r}
        &Replay operations from register \V{r}\\
\V{@@}
        &Repeat last \V{@r} (with optional prefixed repeat count)\\
\V{:reg}
        &Show all current registers
    \end{tabularx}



    \pagebreak



\definecolor{accentcolor}{named}{solarizedmagenta}
\topic{Editor commands}
    \begin{tabularx}{\columnwidth}{l>{\raggedright\arraybackslash}X}
\V{:h subject}
        &Show help text on \emph{subject}\\
\V{:w}, \V{:saveas}
        &Save file; takes optional filename\\
\V{:w >{}> filename}
        &Append text to file\\
\V{:q}
        &Quit editor\\
\V{:wq}, \V{:x}, \V{ZZ}
        &Save file and quit editor\\
\V{:q!}, \V{ZQ}
        &Quit editor without saving changes\\
\V{:e filename}
        &Edit another file (in current buffer)\\
\V{:n}, \V{:N}, \V{:prev}
        &Edit next/previous file in arg list\\
\V{:wn}, \V{:wN}, \V{:wprev}
        &Save file and edit next/prev in arg list\\
\V{:! command}
        &Execute shell command\\
\V{:r file}
        &Insert file contents after current line\\
\V{:r !command}
        &Insert command output after current line\\
\V{:0r file/!cmd}
        &Insert file or command at start of buffer\\
\verb|:$r file/!cmd|
        &Insert file or command at end of buffer\\

\verb|^r^w|
        &Copy word under cursor to command\\
\verb|^f|
        &Edit command line in normal mode\\
\verb|^d|
        &Completion suggestions
    \end{tabularx}

\subtopic{Windows and tabs}
In all commands below, \verb|^w x| may also be typed as \verb|^w^x|.
\vspace{.25\baselineskip}

    \begin{tabularx}{\columnwidth}{l>{\raggedright\arraybackslash}X}
\verb|:split|, \verb|^w s|
        &Split window horizontally\\
\verb|:vsplit|, \verb|^w v|
        &Split window vertically\\
        &(\verb|:split|, \verb|:vsplit| take optional filename)\\
\verb|:new|, \verb|^w n|
        &Split horizontally; new buffer is empty\\
\verb|:vnew|
        &Split vertically; new buffer is empty\\
\verb|^w hjkl|
        &Move to window left/below/above/right \linebreak(with \textit{tmux-navigator.vim:} also \verb|^hjkl|)\\
\verb|^w w|
        &Cycle through open windows\\
\verb|^w HJKL|
        &Move window left/down/up/right\\
\verb|^w rR|
        &Rotate windows clockwise/countercw.\\
\verb|^w +-<>|
        &Increase/decrease window height/width\\
\verb|^w =|
        &Resize windows to equal size\\
\verb|^w q|, \verb|:close|
        &Quit window (\verb|:close| fails if only buffer)\\
\verb|:only|, \verb|^w o|
        &Close all other windows\\

\verb|:term [cmd]|
        &Open terminal and run \verb|cmd| (default: shell)\\
\verb|^\ n|
        &Switch to Terminal-Normal mode\\

\verb|:tabnew|
        &Open new tab; takes optional filename\\
\verb|^w T|
        &Move current window to new tab\\
\verb|:tabclose|
        &Close current tab\\
\verb|:tabonly|
        &Close all other tabs\\
\verb|^<PageDown>|, \verb|gt|
        &Move to next tab\\
\verb|^<PageUp>|, \verb|gT|
        &Move to previous tab
    \end{tabularx}



\definecolor{accentcolor}{named}{solarizedred}
\topic{From my \textit{.vimrc}}
\begin{verbatim}
 set nocompatible    " allow non-vi-compatible stuff
 set encoding=utf-8
 set fileformat=unix
 filetype plugin on
 syntax enable
 set autoindent      " copy indent from prev line
 filetype indent on  " auto-indent based on filetype
 set tabstop=4       " for showing existing tabs
 set softtabstop=4   " used when editing
 set expandtab       " expand tabs to spaces ...
 au BufRead,BufFilePre,BufNewFile Makefile* \
   set noexpandtab   " ... except in Makefiles
 set shiftwidth=4    " used when shifting text
 set shiftround      " shift to multiple of sw
 set smarttab        " auto-indent by sw
 set number          " show line numbers
 set laststatus=2    " show status line
 set showmatch       " show matching quote/paren
 set incsearch       " show matches while typing
 set hlsearch        " highlight all matches
 set virtualedit=block  " select beyond EOL
 set splitbelow      " open new windows below ...
 set splitright      " ... or to the right

 fun! RemoveTrailingWhitespace()
     let l:save = winsaveview()
     keeppatterns %s/\s\+$//e
     call winrestview(l:save)
 endfun
 au BufWritePre * :call RemoveTrailingWhitespace()

set ttimeoutlen=10          " fix timeout after Esc
tnoremap <Esc> <C-\><C-n>   " Terminal-Normal mode

 " Use 24-bit colours and true italics
 if exists('+termguicolors')
     let &t_8f = "\<Esc>[38;2;%lu;%lu;%lum"
     let &t_8b = "\<Esc>[48;2;%lu;%lu;%lum"
     set termguicolors
 endif
 highlight Comment cterm=italic
\end{verbatim}

\columnbreak



\definecolor{accentcolor}{named}{solarizedyellow}
\topic{My favorite plugins}
Plugins go in subdirs of \verb|$HOME/.vim/pack/packages/start/|; run~\V{:helptags ALL} to generate help texts.
\vspace{.5\baselineskip}

    \begin{tabularx}{\columnwidth}{>{\itshape}l>{\raggedright\arraybackslash}X}
autopairs.vim           &Add/delete quotes, parens in pairs\\
commentary.vim          &\V{gc} operator to (un)comment stuff\\
lightline.vim           &Configurable statusline\\
gitbranch.vim           &Show branchname in status line\\
python-pep8-indent.vim  &Fix auto-indentation for Python\\
tmux-navigator.vim      &Switch between vim and tmux\\
nord.vim                &My favourite dark color scheme
    \end{tabularx}

\subtopic{\textit{surround.vim}}
    \begin{tabularx}{\columnwidth}{ll>{\raggedright\arraybackslash}X}
\V{ys arg1 arg2}
        &\multicolumn{2}{>{\raggedright\arraybackslash}X}{Quote or parenthesize something. \linebreak\V{arg1} is a motion, text object etc., \V{arg2}~chooses quotes/parens: \makebox[0pt][l]{\V{"} \V{]} \V{)} \V{\}} \V{>} \V{<}} \linebreak Args \V{[} \V{(} \V{\{} add space inside parens.}\\
&\V{ysM>}
        &wraps \V{M} in \V{<} and \V{>}\\
&\V{ysM<}
        &wraps \V{M} in \V{<...>} and \V{</...>} tags (prompts for tag name); same as \V{t}\\
&\V{ysMf}
        &wraps \V{M} in parentheses and prepends function call (prompts for name); \V{F}~adds~extra space inside parentheses\\
&\V{ysM\textbackslash}
        &wraps \V{M} in LaTeX \V{\textbackslash begin}-\V{\textbackslash end} block (prompts for environment name)\\
&\V{ysM*}
        &wraps \V{M} in \V{*} (like \V{*M*}); also \verb+_+, \verb+|+, \verb+#+ etc.\\
\V{S arg2}
        &\multicolumn{2}{>{\raggedright\arraybackslash}X}{Use in Visual mode; \texttt{arg1} is implicit}\\
\V{cs old new}
        &\multicolumn{2}{>{\raggedright\arraybackslash}X}{Change surrouding quotes or parens:}\\
&\V{cs'"}
        &changes single to double quotes\\
&\V{cs)\}}
        &changes parentheses to curly braces\\
&\V{cs)(}
        &adds space inside parentheses\\
\V{ds arg}
        &\multicolumn{2}{>{\raggedright\arraybackslash}X}{Delete surrounding quotes, parens}
    \end{tabularx}


\subtopic{\textit{fugitive.vim}}
    \begin{tabularx}{\columnwidth}{ll>{\raggedright\arraybackslash}X}
\V{:G}
        &\multicolumn{2}{>{\raggedright\arraybackslash}X}{Show Git status. Key maps (see also \V{g?}):}\\
&\V{gu}, \V{gU}, \V{gs}
        &to (\textit{n}th) untracked/unstaged/staged file\\
&\V{<}, \V{>}, \V{=}
        &show/hide/toggle in-line diff\\
&\V{(}, \V{)}
        &to previous/next file or chunk\\
&\V{s}, \V{u}
        &stage/unstage current file or chunk\\
&\V{cc}
        &commit\\
&\V{ca}, \V{ce}, \V{cw}
        &amend/-\strut-no-edit/reword last commit\\
&\V{gq}
        &quit status buffer\\
\V{:G cmd}
        &\multicolumn{2}{>{\raggedright\arraybackslash}X}{Run \texttt{git cmd}}
    \end{tabularx}



    {
        \color{solarizedbase00}\vfill\hrule\par%
        % {\noindent\rule{0pt}{8pt}\footnotesize\hfill Copyright 2020–2023 drs-p (\url{https://github.com/drs-p}), {\addfontfeature{Letters = UppercaseSmallCaps} CC-BY-SA 4.0}\hbox{}}\par
        {\noindent\rule{0pt}{8pt}\footnotesize © 2020-2023 \href{https://github.com/drs-p/}{drs-p}\hfill released under the terms of the \href{https://creativecommons.org/licenses/by-sa/4.0/}{{\addfontfeature{Letters = UppercaseSmallCaps}CC-BY-SA 4.0}} license}\par
    }

\end{multicols*}
\end{document}
